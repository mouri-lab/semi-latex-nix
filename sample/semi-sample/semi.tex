% \documentclass[submit,techreq,noauthor]{eco}	% ゼミ用テンプレート
\documentclass[submit,techreq,noauthor,dvipdfmx]{b3-eco}	% b3用テンプレート
% \documentclass[submit,techreq,noauthor,dvipdfmx]{mid-eco}	% b4中間発表テンプレート

\usepackage[dvipdfmx]{graphicx}
% \usepackage{mediabb}				% for pdf include
\usepackage{listings, jlisting} 		% for source code
\usepackage{url}
\usepackage{setspace}
\usepackage{enumitem}
\usepackage{booktabs}
\usepackage[inkscapepath=./.svg-inkscape/]{svg}

% フォントの警告を無視
\usepackage{silence}
\WarningFilter{latexfont}{Some font shapes}
\WarningFilter{latexfont}{Font shape}

% listingsの設定
\lstset{
	%プログラム言語(複数の言語に対応,C,C++も可)
 	% language = C++,
 	%枠外に行った時の自動改行
 	breaklines = true,
 	%自動改行後のインデント量(デフォルトでは20[pt])
 	breakindent = 10pt,
 	%標準の書体
 	basicstyle = \ttfamily\scriptsize,
 	%コメントの書体
 	commentstyle = {\itshape \color[cmyk]{1,0.4,1,0}},
 	%関数名等の色の設定
 	% classoffset = 0,
 	%キーワード(int, ifなど)の書体
 	% keywordstyle = {\bfseries \color[cmyk]{0,1,0,0}},
 	%表示する文字の書体
 	stringstyle = {\ttfamily \color[rgb]{0,0,1}},
 	%枠 "t"は上に線を記載, "T"は上に二重線を記載
	%他オプション:leftline,topline,bottomline,lines,single,shadowbox
 	frame = ltbr,
	%行番号の位置
	% numbers = left,
	%行番号の間隔
 	stepnumber = 1,
	%タブの大きさ
 	tabsize = 4,
}

% \setstretch{1.5} % 行間を広くします(資料チェックしてもらうときはコメントを外す)

\begin{document}

\semino {6/1}					% 年度/回数
\date   {6/4/1/火}				% 令和/月/日/曜日
\title  {地球環境に配慮した毛利研究室ゼミテンプレート}	% タイトル
\author {立命 太郎}				% 氏名


\begin{abstract}
	近年,地球の環境破壊が問題となっている.
	限りある資源を有効に活用するため,ペーパレスを推進する企業も増え始めている.
	我が毛利研究室でも,ゼミ資料の紙の使用量を抑制する動きが見られている.
	そこで,表紙を無くした新たなゼミテンプレートの作成を行った.
	本稿では,新しいゼミテンプレートと付録として添付したMakefileの使用法について述べる.
\end{abstract}
\maketitle

%%%%%%%%%%%%%%%%%%%%%ここから消して下さい%%%%%%%%%%%%%%%%%%%%%
\section{はじめに}
本稿では,
だが
つまり
そして
かもしれない
テンプレートのディレクトリ構成と
Makefileの概要や画像の挿入方法,参考文献の書き方について述べる.
資料をチェックしてもらうときは,本ファイル8行目の
\verb|\setstretch{1.5}|
のコメントを外すとチェックする側はありがたいです.
LaTexのコンパイル時にエラーが出るor文字化けする場合は,文字コードが原因の可能性が高いです.
テンプレートはUTF-8にしていますが,各自環境に合わせて設定して下さい.


\section{ディレクトリ構成}
本プロジェクトのディレクトリ構成は以下のようになっている.
共通のスタイルファイルは \texttt{style/} ディレクトリに集約されており,各ドキュメントは \texttt{sample/} 以下のサブディレクトリに配置される.

\begin{lstlisting}
├─ Makefile        % ビルド用スクリプト
├─ style/          % 共通スタイルファイル
│   ├── b3-eco.cls
│   ├── eco.cls
│   └── ...
└─ sample/
    └─ semi-sample/    % 本ドキュメントのディレクトリ
        ├── fig/       % 画像ファイル
        ├── references.bib % 参考文献データベース
        └── semi.tex   % 本文ソースコード
\end{lstlisting}

\section{Makefileの使用法}
ビルドにはプロジェクトルートにある \texttt{Makefile} を使用する.
以下のコマンドで PDF を生成できる.

\begin{lstlisting}
$ make build sample/semi-sample
\end{lstlisting}

生成された PDF は \texttt{sample/semi-sample/semi.pdf} として出力される.
中間ファイルを削除するには以下のコマンドを実行する.

\begin{lstlisting}
$ make clean sample/semi-sample
\end{lstlisting}

ファイルの変更を監視して自動ビルドするには以下のコマンドを実行する.

\begin{lstlisting}
$ make watch sample/semi-sample
\end{lstlisting}


\section{参考文献の書き方}
参考文献は,BibTeXを使う.
たとえば,図\ref{fig:bibsample}の内容を含むファイル(references.bib)を作り,
\verb|\cite{etx}|の様に本文で参照\cite{etx}し,pbibtexコマンドで参考文献リストを作成します.
論文データベースには,必ずbibtex形式というのが用意されているはず.
その内容をコピーすれば基本は大丈夫なはず(必ずチェックする).
参考文献のスタイルは,情報処理学会の出現順のものを使用しています.

\begin{figure*}[t]
	\begin{lstlisting}
@INPROCEEDINGS{etx,
  author = {Douglas S. J. De Couto and Daniel Aguayo and John C. Bicket and Robert Morris},
  title = {A high-throughput path metric for multi-hop wireless routing},
  booktitle = {Proc. of ACM MobiCom '03},
  year = {2003},
  pages = {134-146}
}
    \end{lstlisting}
	\vspace{-2mm}
	\caption{BibTeXの記述例}
	\label{fig:bibsample}
\end{figure*}



\section{図の挿入方法}
\subsection{TgifやOpenOfficeで作る場合}
TgifやOpenOfficeで作る場合は,epsで出力して,includegraphicsで挿入しましょう(例:図\ref{fig:tgif-sample}).

\begin{figure}[t]
	\centering
	\includegraphics[width=8cm]{fig/ex1.eps}
	\caption{Open Officeで作成した図}
	\label{fig:tgif-sample}
\end{figure}

\subsection{PowerPointで作る場合}
複雑な図を作るときは,Microsoft PowerPointやVisioがおすすめ.
図をPDFでエクスポートし,それをTeXで表示できます(例: 図\ref{fig:pdf-sample}).
PDFを作成時にフォントが埋め込まれているかを確認する.
場合によっては,図のフォントが文字化けすることがあるので注意.

\begin{figure}[t]
	\centering
	\includegraphics[width=8cm]{fig/ex2.pdf}
	\caption{PowerPointで作成した図}
	\label{fig:pdf-sample}
\end{figure}


\subsection{svgファイル}
svgを貼るときはincludesvgコマンドを使います(例: 図\ref{fig:svg-sample}).

\begin{figure}[t]
	\centering\includesvg[width=80mm, angle=0]{fig/svg-sample.svg}
	\caption{svgファイル}\label{fig:svg-sample}
\end{figure}



%%%%%%%%%%%%%%%%%%%%%ここまで消して下さい%%%%%%%%%%%%%%%%%%%%%

%bibtex
\setlength\baselineskip{12pt}
{\small
	\bibliography{references}
	\bibliographystyle{ipsjunsrt}
}


\end{document}